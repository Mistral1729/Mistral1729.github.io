\documentclass[11pt]{scrreprt}
%\usepackage[pdftex]{hyperref}
%\usepackage[numbers]{natbib}
\usepackage{graphicx}
\usepackage[ngerman]{babel}
%\usepackage[ansinew]{inputenc} %Umlaute
\usepackage[utf8]{inputenc}     %Umlaute
\usepackage{amssymb}
\usepackage{bbm}
\usepackage{amsmath}
\usepackage{amsthm}
%\usepackage{cases}
%\usepackage{booktabs}
%\usepackage{eufrak}
%\usepackage{fullpage}




\renewcommand{\P}{\mathbb {P}} %Wahrscheinlichkeit
\newcommand{\E}{\mathbb {E}}   %Erwartungswert
\newcommand{\C}{\mathbb {C}}   %Komplexe Zahlen
\newcommand{\R}{\mathbb {R}}   %Reelle Zahlen
\newcommand{\Q}{\mathbb {Q}}   %Rationale Zahlen
\newcommand{\Z}{\mathbb {Z}}   %Ganze Zahlen
\newcommand{\N}{\mathbb {N}}   %Natuerliche Zahlen
\newcommand{\ind}{\mathbbm{1}} %Indikatorfunktion
\newcommand{\eps}{\varepsilon} %Für epsilon benutzen
\newcommand{\bsl}{\backslash}  %Für backslash
\newcommand{\dif}{d}
\newcommand{\AAA}{\mathcal {A}}  %\sigma-Algebra
\newcommand{\FFF}{\mathcal {F}} %Algebra
\newcommand{\EEE}{\mathcal {E}}
\newcommand{\GGG}{\mathcal {G}}
\newcommand{\BBB}{\mathcal {B}} %Borel-\sigma-Algebra
\newcommand{\notsim}{\nsim}



\renewcommand{\Re}{\operatorname{Re}}
\renewcommand{\Im}{\operatorname{Im}}


\DeclareMathOperator{\supp}{supp}
\DeclareMathOperator{\Var}{Var}
\DeclareMathOperator{\Cov}{Cov}
\DeclareMathOperator{\Corr}{Corr}
\DeclareMathOperator{\const}{const}
\DeclareMathOperator{\argmax}{argmax}
\DeclareMathOperator{\sgn}{sgn}


%%%%%%%%%%%%%%%%%%%%%%%%%%%%%%%%%%%%%%%%%%%%%%%%%%%%%%%%%%%%%%%
%             Konvergenzarten                                 %
%%%%%%%%%%%%%%%%%%%%%%%%%%%%%%%%%%%%%%%%%%%%%%%%%%%%%%%%%%%%%%%
\newcommand{\eqdistr}{\stackrel{{d}}{=}}
\newcommand{\eqdef}{\stackrel{{def}}{=}}
\newcommand{\toweak}{\overset{{w}}{\underset{n\to\infty}\longrightarrow}}
\newcommand{\todistr}{\overset{{d}}{\underset{n\to\infty}\longrightarrow}}
\newcommand{\tod}{\overset{{d}}{\underset{n\to\infty}\longrightarrow}}
\newcommand{\toprobab}{\overset{{P}}{\underset{n\to\infty}\longrightarrow}}
\newcommand{\tofd}{\overset{{f.d.d.}}{\underset{n\to\infty}\longrightarrow}}
\newcommand{\tofs}{\overset{{f.s.}}{\underset{n\to\infty}\longrightarrow}}
\newcommand{\toas}{\overset{{a.s.}}{\underset{n\to\infty}\longrightarrow}}
\newcommand{\tosquare}{\stackrel{L^2}{\to}}
\newcommand{\ton}{\overset{}{\underset{n\to\infty}\longrightarrow}}
\newcommand{\toLp}{\overset{{L^p}}{\underset{n\to\infty}\longrightarrow}}
\newcommand{\toLs}{\overset{{L^s}}{\underset{n\to\infty}\longrightarrow}}



%%%%%%%%%%%%%%%%%%%%%%%%%%%%%%%%%%%%%%%%%%%%%%%%%%%%%%%%%%%%%%%
%             Verteilungen                                    %
%%%%%%%%%%%%%%%%%%%%%%%%%%%%%%%%%%%%%%%%%%%%%%%%%%%%%%%%%%%%%%%
\def\Norm {{\mathrm{N}}}   %Normalveretreilung
\def\Poi {{\mathrm{Poi}}}  %Poisson-Verteilung
\def\Geo {{\mathrm{Geo}}}  %Geometrische Verteilung
\def\Exp {{\mathrm{Exp}}}  %Exponentialverteilung
\def\Bin {{\mathrm{Bin}}}  %Binomialverteilung
\def\NBin {{\mathrm{NB}}}  %Negative Binomialverteilung
\newcommand{\U}{\mathrm {U}}   %Gleichverteilung



%%%%%%%%%%%%%%%%%%%%%%%%%%%%%%%%%%%%%%%%%%%%%%%%%%%%%%%%%%%%%%%
%             Beweis                                          %
%%%%%%%%%%%%%%%%%%%%%%%%%%%%%%%%%%%%%%%%%%%%%%%%%%%%%%%%%%%%%%%

\newcommand{\bew}{\textsc{Beweis.} }            %Beweis
\newcommand{\bewn}[1]{\textsc{#1.} }
\newcommand{\ebew}{\hfill \qed \vspace*{5mm}}   %Ende des Beweises



\setlength{\parindent}{0pt}



%%%%%%%%%%%%%%%%%%%%%%%%%%%%%%%%%%%%%%%%%%%%%%%%%%%%%%%%%%%%%%%
%             Satz, Lemma, Definition, ...                    %
%%%%%%%%%%%%%%%%%%%%%%%%%%%%%%%%%%%%%%%%%%%%%%%%%%%%%%%%%%%%%%%
\theoremstyle{plain}
\newtheorem{satz}{Satz}[chapter]
\newtheorem{beh}[satz]{Behauptung}
\newtheorem{kor}[satz]{Korollar}
\newtheorem{lem}[satz]{Lemma}
\newtheorem{prop}[satz]{Proposition}

\theoremstyle{remark}
\newtheorem{bem}[satz]{Bemerkung}
\newtheorem{bsp}[satz]{Beispiel}
\newtheorem{ueb}[satz]{Übung}
\newtheorem{dfn}[satz]{Definition}




\numberwithin{equation}{chapter}
\numberwithin{satz}{chapter}
\numberwithin{section}{chapter}




\begin{document}

\begin{titlepage}

\includegraphics[width=12cm]{WWU_Logo.jpg}

\vspace{6cm}

\begin{center}
{\huge \scshape Der zentrale Grenzwertsatz}\\
\vspace{7mm}
{\Large \textsc{Bachelorarbeit}}\\
\end{center}

\vspace{7cm}

\begin{tabular}{ll}
Eingereicht von:  &Ihr Name\\
Datum:            &\\
Erstgutachter:    &Name\\
Zweitgutachter:   &Name\\
\end{tabular}

\vspace{3cm}
\end{titlepage}


\tableofcontents



\chapter{Der zentrale Grenzwertsatz}
\section{Formulierung des zentralen Grenzwertsatzes}

\begin{satz}[Der zentrale Grenzwertsatz]\label{satz:zgws}
Seien $X_1, X_2, \ldots$ unabhängige, identisch verteilte, quadratisch integrierbare Zufallsvariablen mit $\E X_k = \mu$, $\Var X_k = \sigma^2\in (0,\infty)$. Sei $S_n = X_1 + \ldots + X_n$. Dann gilt
$$
\frac{S_n - n\mu}{\sigma \sqrt{n}} \tod N,
$$
wobei $N \sim N(0,1)$ eine standardnormalverteilte Zufallsvariable ist.
\end{satz}
\begin{bem}
Mit anderen Worten:  Für alle $x \in \R$ gilt
$$
\lim_{n \to \infty} \P\left[\frac{S_n - n\mu}{\sigma \sqrt{n}} \leq x \right] = \Phi(x)
\text{ mit }
\Phi(x) = \frac{1}{\sqrt{2 \pi}} \int\limits_{-\infty}^x  e^{-\frac{t^2}{2}} \dif t.
$$
\end{bem}

%\begin{bem}
%Für großes $n$ gilt: $S_n \sim N(n\mu, \sigma^2 n)$.
%\end{bem}

Als Spezialfall von Satz~\ref{satz:zgws} erhalten wir den folgenden Satz von de Moivre--Laplace.

\begin{kor}[de Moivre (1733), Laplace (1812)]\label{satz:de_Moivre}
Sei $S_n \sim \Bin(n,p)$ binomialverteilt, wobei $p \in (0,1)$ konstant sei. Dann gilt für alle $x \in \R$, dass
$$
\lim_{n \to \infty} \P\left[ \frac{S_n - np}{\sqrt{np (1-p)}} \leq x \right] = \Phi(x).
$$
\end{kor}

\bew
Wir definieren unabhägige, identisch verteilte Zufallsvariablen $X_1, X_2, \ldots$ mit
$$
 \P[X_k = 1] = p \text{ und } \P[X_k = 0] = 1-p.
$$
Dann gilt für die Summe der Zufallsvariablen
$$
S_n = X_1 + \ldots + X_n \sim \Bin(n,p).
$$
Der Erwartungswert von $X_k$ ist $\mu = p$ und die Varianz ist $\sigma^2 = p \cdot (1-p)$. Mit dem zentralen Grenzwertsatz folgt die Behauptung.
\ebew

\section{Beweis des zentralen Grenzwertsatzes}
\bewn{Beweis von Satz~\ref{satz:zgws}}
Der nachfolgende Beweis ist an die Darslellung in den Büchern von Feller~\cite{Feller_buch} und Kallenberg~\cite{Kallenberg_buch} angelehnt.
$$
\ldots \ldots \ldots
$$
\ebew


\chapter{Ein weiteres Kapitel}
\section{Eine weitere Sektion}
Eine Gleichung ohne Nummer:
$$
e^{\pi i} = -1.
$$
Eine nummerierte Gleichung
\begin{equation}\label{eq:euler}
e^{\pi i} = -1.
\end{equation}
Aus~\eqref{eq:euler} folgt durch Quadrieren, dass $e^{2\pi i} = 1$.

\begin{satz}
Text...
\end{satz}

\begin{lem}
Text...
\end{lem}

\begin{bsp}
Text...
\end{bsp}

\section{Noch eine Sektion}
\begin{satz}
Text...
\end{satz}


\begin{thebibliography}{100}

\bibitem{Kallenberg_buch}
Kallenberg, O.: {\em Foundations of Modern Probability}. Second Edition, Springer (2002).

\bibitem{Feller_buch}
Feller, W.:  {\em An Introduction to Probability Theory and Its Applications}. Third Edition, Wiley (1968).

\end{thebibliography}

\end{document}
